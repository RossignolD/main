\usepackage[subpreambles=true]{standalone}
% \usepackage[charter]{mathdesign} % requires package ly1 for texnansi.enc
\usepackage[notext]{stix2} % provides \sqsubsetneq
\usepackage{mathtools}	% revised version of amsmath
\usepackage{tipa}
\usepackage[amsmath,hyperref,thmmarks]{ntheorem}
\usepackage[skins,breakable]{tcolorbox} 
\usepackage{tikz}
\usepackage[linguistics]{forest}
\usepackage{etoolbox}
\usepackage{environ}
\usepackage{fancyhdr}
\fancyhead[L]{Thomas Graf, Fall 2021}
\fancyhead[R]{\leftmark}
\pagestyle{fancy}


\raggedbottom

\newcounter{exercise}
\def\theexercise{\arabic{exercise}}

\newenvironment{exercise}[1][\unskip]{%
    \refstepcounter{exercise}%
    \bigskip%
    \noindent\ignorespaces%
    \begin{tcolorbox}[
      blanker,
      width=0.85\textwidth,enlarge left by=0.15\textwidth,
      before skip=6pt,
      breakable,
      overlay unbroken and first={%
        \node[inner sep=0pt,outer sep=0pt,text width=0.15\textwidth,
          align=left,
          below right]
          at ([xshift=-0.15\textwidth]frame.north west)
      {%
        \noindent%
        \bfseries\color[HTML]{2D68A2}
        #1%
        Exercise\\
        \smallskip
        \LARGE\theexercise%
      };}]%
}%
{%
    \end{tcolorbox}%
    % \smallskip\noindent%
}

\providebool{showanswers}
\newcommand{\exranswerlist}{}
\NewEnviron{exranswer}{\ifbool{showanswers}{%
  \xappto\exranswerlist{\theexercise . \BODY}}{}}

\newcounter{example}
\def\theexample{\arabic{example}}

\newenvironment{example}[1][\unskip]{%
    \refstepcounter{example}%
    \bigskip%
    \noindent\ignorespaces%
    \begin{tcolorbox}[
      blanker,
      width=0.85\textwidth,enlarge left by=0.15\textwidth,
      before skip=6pt,
      breakable,
      overlay unbroken and first={%
        \node[inner sep=0pt,outer sep=0pt,text width=0.15\textwidth,
          align=left,
          below right]
          at ([xshift=-0.15\textwidth]frame.north west)
      {%
        \noindent%
        \bfseries\color[HTML]{1A933D}%
        #1%
        Example\\
        \smallskip
        \LARGE\theexample%
      };}]%
}
{%
    \end{tcolorbox}%
    % \smallskip\par\noindent%
}

\newenvironment{advice}[1][\unskip]{%
    \bigskip%
    \noindent\ignorespaces%
    \begin{tcolorbox}[
      blanker,
      width=0.85\textwidth,enlarge left by=0.15\textwidth,
      before skip=6pt,
      breakable,
      overlay unbroken and first={%
        \node[inner sep=0pt,outer sep=0pt,text width=0.15\textwidth,
          align=left,
          below right]
          at ([xshift=-0.15\textwidth]frame.north west)
      {%
        \noindent%
        \bfseries\color[HTML]{D425A7}%
        #1%
        Advice
      };}]%
}
{%
    \end{tcolorbox}%
    % \smallskip\par\noindent%
}

%     \begin{minipage}[t]{.15\textwidth}
%         \noindent%
%         \bfseries\color[HTML]{1A933D}%
%         #1%
%         Example\\
%         \LARGE\theexample%
%     \end{minipage}
%     \begin{minipage}[t]{.85\textwidth}%
% }%
% {%
%     \end{minipage}%
%     \medskip\par\noindent\ignorespacesafterend%
% }


\theoremstyle{plain}
\theoremheaderfont{\normalfont\bfseries}
\theorembodyfont{\normalfont\upshape}
\theoremseparator{.}
\newcommand{\TheoremSymbol}{\tikz{\draw (0,0) -| (.5em,.5em);}}
\theoremsymbol{\ensuremath{\TheoremSymbol}}
\newtheorem{theorem}{Theorem}
\newtheorem{lemma}[theorem]{Lemma}
\newtheorem{proposition}[theorem]{Proposition}
\newtheorem{corollary}[theorem]{Corollary}

\theorembodyfont{\normalfont\itshape}
\theoremsymbol{}
\newtheorem{conjecture}[theorem]{Conjecture}
\newtheorem{problem}[theorem]{Problem}

\theorembodyfont{\normalfont\upshape}
\theoremprework{\bigskip\hrule}
\theorempostwork{\hrule\bigskip}
\newtheorem{definition}[theorem]{Definition}

\theoremstyle{nonumberplain}
\theoremheaderfont{\normalfont\itshape}
\theoremprework{}
\theorempostwork{}
\newcommand{\myQED}{\ensuremath{\blacksquare}} % avoid clash with stix
\qedsymbol{\myQED}
\theoremsymbol{\ensuremath{\Box}}
\newtheorem{proof}{Proof}

% \theoremsymbol{\ensuremath{\bigcirc}}
\theoremsymbol{}
\newtheorem{remark}{Remark}
% \newtheorem{example}{Example}

\theoremsymbol{}
\theoremheaderfont{\normalfont\scshape}
\theoremindent1.5em
\newtheorem{ucase}{Case}

\theoremstyle{plain}
\newtheorem{case}{Case}
\theoremindent1.5em
\newtheorem{subcase}{Case}
\numberwithin{subcase}{case}
